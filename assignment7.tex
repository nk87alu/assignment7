\documentclass{article}
\usepackage{pawlowski}
\usepackage{amsmath}
\usepackage{natbib}
\usepackage{graphicx}

\begin{document}
\section{Task One}

We have used multiple methods to find derivatives numerically. These include: 
\begin{enumerate}
    \item Euler's Method
    \item Central Differencing 
\end{enumerate} 

 Euler's method uses an iterative process where each new point is used to find the next one in order to solve ordinary differential equations. It is given by \begin{align}
y_{n+1}=y_n+h*f(x_n,y_n).  
\end{align} 
 Central differencing uses the second order terms from two different points for the Taylor series to approximate derivatives. It is still simple to implement, but can have less inherent error due to using and canceling more terms of the series. Below is a derivation of the function used in central differencing. 
\begin{align}
f(x+h)=f(x)+hf^`(x)+\frac{h^2}{2}f^{``}(x)+... \\
f(x-h)=f(x)-hf^`(x)+\frac{h^2}{2}f^{``}(x)-...\\
f(x+h)-f(x-h)=(f(x)+hf^`(x)+...)-(f(x)-hf^`(x)+...)\\
f(x+h)-f(x-h)=2hf^`(x)\\
f^`(x)=\frac{f(x+h)-f(x-h)}{2h}
\end{align}

\section{Task Two}

For task two, previously created code was modified to accommodate 50 random point charges on a grid. Modifications included importing qgrid, expanding the dimensions of our grid, trimmed extreme potential values to allow for better color visualization of charges and added colored dots to make it easier to see the locations of the point charges. The stream plot, when 50 charges are used seems to become less useful, but still better looking overall than a vector plot. Fig. 1 shows a plot generated by our code for 50 charges. 
\begin{figure}
    \centering
    \includegraphics[width=0.5\linewidth]{image.png}
    \caption{Potential and Electric Field plot for 50 Point Charges}
    \label{fig:1}
\end{figure}

\section{Task Three}
In task three, Euler's method for solving ODE's was implemented to plot position vs time for a harmonic oscillator. 
\subsection{C=0}
For C=0, we observed a continuous oscillating pattern. This makes sense as this case zero damping. Fig. 2 shows C=0. 
\begin{figure}
    \centering
    \includegraphics[width=0.5\linewidth]{c0.png}
    \caption{Motion of a harmonic oscillator for C=0}
    \label{fig:2}
\end{figure}
\subsection{C=1}
C=1 provides an under damped system. This lack of damping allows around one complete oscillation before returning to a resting state. 
\begin{figure}
    \centering
    \includegraphics[width=0.5\linewidth]{c1.png}
    \caption{Motion of a harmonic oscillator for C=1}
    \label{fig:3}
\end{figure}
\subsection{C=2}
For C=2, it can be seen in Fig. 4 that this essentially eliminates the oscillation with only a very minor 'bump' seen at around t=5 (s).
\begin{figure}
    \centering
    \includegraphics[width=0.5\linewidth]{c2.png}
    \caption{Motion of a harmonic oscillator for C=2}
    \label{fig:4}
\end{figure}
\subsection{C=3}
Finally, C=3 provided an over damped system. This system results in zero oscillation whatsoever. Fig. 5 clearly shows an immediate and smooth drop to a stable state. 
\begin{figure}
    \centering
    \includegraphics[width=0.5\linewidth]{c3.png}
    \caption{Motion of a harmonic oscillator for C=3}
    \label{fig:5}
\end{figure}
\section{Task Four}
A cyclists motion without drag was modeled using Euler's method for ODE's and python. Fig. 6 shows the smooth acceleration from rest up until it begins to flatten off, which may be counterintuitive since there is no drag. This leveling off is due to the P/(mv) term getting smaller as v increases, and in turn dv/dt gets smaller so the acceleration is smaller. 
\begin{figure}
    \centering
    \includegraphics[width=0.5\linewidth]{bike.png}
    \caption{Plot of a cyclists Velocity Vs Time without drag}
    \label{fig:bike}
\end{figure}
\end{document}
 